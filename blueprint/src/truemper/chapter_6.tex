\section{Chapter 6}

\subsection{Chapter 6.2}

Goal of the chapter: separation algorithm for deciding if there exists a separation of a matroid induced by a separation of its minor.

\begin{proposition}[6.2.1]
  \label{prop:6.2.1}
  \uses{def:k_sep,def:minor,def:binary_matroid}
  Partitioned version of matrix $B^{N}$ representing a minor $N$ of a binary matroid $M$, where $N$ has an exact $k$-separation for some $k \geq 1$.
\end{proposition}

\begin{proposition}[6.2.3]
  \label{prop:6.2.3}
  \uses{prop:6.2.1}
  Matrix $B$ for $M$ displaying partitioned $B^{N}$
\end{proposition}

\begin{proposition}[6.2.5]
  \label{prop:6.2.5}
  \uses{prop:6.2.1,prop:6.2.3}
  Matrix $B$ for $M$ with partitioned $B^{N}$, row $x \in X_{3}$, and column $y \in Y_{3}$.
\end{proposition}

\begin{lemma}[6.2.6]
  \label{lem:6.2.6}
  \uses{def:binary_matroid,def:k_conn,def:1_elem_ext,def:2_elem_ext,def:3_elem_ext,def:loop,def:coloop,def:parallel_elems,def:series_elems}
  Let $N$ be a $3$-connected binary matroid on at least $6$ elements. Suppose a $1$-, $2$-, or $3$-element binary extension of $N$, say $M$, has no loops, coloops, parallel elements, or series elements. Then $M$ is $3$-connected.
\end{lemma}

\begin{proof}[Proof sketch]
  \uses{lem:3.3.20}
  \begin{itemize}
    \item Let $C$ be a binary representation matrix of $M$ that displays a binary representation matrix $B$ for $N$.
    \item By assumption, $B$ is $3$-connected.
    \item $C$ is connected, as otherwise by case analysis $C$ contains a zero vector or unit vector, so $M$ has a loop, coloop, parallel elements, or series elements, a contradiction.
    \item If $C$ is not $3$-connected, then by Lemma 3.3.20 there is a $2$-separation of $C$ with at least $5$ rows/columns on each side. Then $B$ has a $2$-separation with at least $2$ rows/columns on each side, a contradiction.
    \item Thus, $C$ is $3$-connected, so $M$ is $3$-connected.
  \end{itemize}
\end{proof}


\subsection{Chapter 6.3}

\begin{definition}[6.3.2]
  \label{def:6.3.2}
  \uses{def:k_sep,def:binary_matroid,def:minor,def:isomorphism}
  $M$ is called minimal if it satisfies the following conditions.
  \begin{itemize}
    \item $M$ has an $N$ minor.
    \item $M$ has no $k$-separation induced by the exact $k$-separation $(F_{1}, F_{2})$ of $N$.
    \item The matroid $M$ is minimal with respect to the above conditions.
  \end{itemize}
\end{definition}

\begin{definition}[6.3.3]
  \label{def:6.3.3}
  \uses{def:k_sep,def:binary_matroid,def:minor,def:isomorphism}
  $M$ is called minimal under isomorphism if it satisfies the following conditions.
  \begin{itemize}
    \item $M$ has at least one $N$ minor.
    \item Some $k$-separation of at least one such minor corresponding to the exact $k$-separation $(F_{1}, F_{2})$ of $N$ under one of the isomorphisms fails to induce a $k$-separation of $M$.
    \item The matroid $M$ is minimal with respect to the above conditions.
  \end{itemize}
\end{definition}

\begin{proposition}[6.3.11]
  \label{prop:6.3.11}
  \uses{def:6.3.2,prop:6.2.5}
  Matrix $B$ for $M$ with partitioned $B^{N}$, row $x \in X_{3}$, and column $y \in Y_{3}$.
\end{proposition}

\begin{proposition}[6.3.12]
  \label{prop:6.3.12}
  \uses{def:6.3.2,prop:6.2.1,prop:6.2.5}
  Partitioned version of $B^{N}$: $B^{N} = \mathrm{diag}(A^{1}, A^{2})$.
\end{proposition}

\begin{definition}[separation algorithm]
  \label{def:separation_algorithm}
  Polynomial-time recursive procedure to search for an induced partition. Described on pages 132--133 and again on pages 137--138.
\end{definition}

\begin{proposition}[6.3.13]
  \label{prop:6.3.13}
  \uses{def:6.3.2,prop:6.3.11,def:separation_algorithm}
  Special case where $B$ of a minimal $M$ contains just one row $x$ beyond $B^{N}$. This proposition gives properties of row subvectors of row $x$ by step $1$ of the separation algorithm.
\end{proposition}

\begin{proposition}[6.3.14]
  \label{prop:6.3.14}
  \uses{def:6.3.2,prop:6.3.11,def:separation_algorithm}
  Special case where $B$ of a minimal $M$ contains just one column $y$ beyond $B^{N}$. This proposition gives properties of column subvectors of column $y$ by step $1$ of the separation algorithm.
\end{proposition}

\begin{lemma}[6.3.15]
  \label{lem:6.3.15}
  \uses{def:6.3.2,prop:6.3.11,def:separation_algorithm}
  Treats the case where $B$ has at least two additional rows or columns beyond those of $B^{N}$.
\end{lemma}

\begin{proof}[Proof sketch]
  \uses{prop:6.3.13,prop:6.3.14}
  Argue about the structure of the matrix, applying steps 1 and 2 of the separation algorithm.
\end{proof}

\begin{lemma}[6.3.16]
  \label{lem:6.3.16}
  \uses{def:6.3.2,prop:6.3.11,lem:6.3.15,def:separation_algorithm}
  Expands case (i) of Lemma 6.3.15.
\end{lemma}

\begin{proof}[Proof sketch]
  Further arguments about the structure of the matrix.
\end{proof}

\begin{lemma}[6.3.17]
  \label{lem:6.3.17}
  \uses{def:6.3.2,prop:6.3.11,lem:6.3.15,def:separation_algorithm}
  Expands case (ii) of Lemma 6.3.15.
\end{lemma}

\begin{proof}[Proof sketch]
  Further arguments about the structure of the matrix.
\end{proof}

\begin{theorem}[6.3.18]
  \label{thm:6.3.18}
  \uses{def:6.3.2,prop:6.3.11}
  Structural description of representation matrix (6.3.11) of a minimal $M$. Contains cases (a), (b), and (c) with sub-cases (c.1) and (c.2).
\end{theorem}

\begin{proof}[Proof sketch]
  \uses{def:6.3.2,prop:6.3.13,prop:6.3.14,lem:6.3.15,lem:6.3.16,lem:6.3.17}
  \begin{itemize}
    \item (6.3.13) and (6.3.14) establish (a) and (b).
    \item Lemmas 6.3.15, 6.3.16, and 6.3.17 prove (c.1) and (c.2).
  \end{itemize}
\end{proof}

\begin{lemma}[6.3.19]
  \label{lem:6.3.19}
  \uses{def:6.3.3,thm:6.3.18}
  Additional structural statements for cases (c.1) and (c.2) of Theorem 6.3.18.
\end{lemma}

\begin{proof}[Proof sketch]
  \uses{thm:6.3.18,lem:6.3.15}
  Reason about representation matrices using Theorem 6.3.18, Lemma 6.3.15, minimality, isomorphisms, pivots, and so on.
\end{proof}

\begin{proposition}[6.3.21]
  \label{prop:6.3.21}
  \uses{def:6.3.3}
  Matrix $B$ for $M$ minimal under isomorphism, case (a).
\end{proposition}

\begin{proposition}[6.3.22]
  \label{prop:6.3.22}
  \uses{def:6.3.3}
  Matrix $B$ for $M$ minimal under isomorphism, case (b).
\end{proposition}

\begin{proposition}[6.3.23]
  \label{prop:6.3.23}
  \uses{def:6.3.3}
  Matrix $\overline{B}$ for minor $\overline{M}$ of $M$ minimal under isomorphism.
\end{proposition}

\begin{theorem}[6.3.20]
  \label{thm:6.3.20}
  \uses{def:6.3.3,prop:6.3.21,prop:6.3.22,prop:6.3.23}
  Let $M$ be minimal under isomorphism. Then one of $3$ cases holds for matrix representation of $M$.
\end{theorem}

\begin{proof}[Proof sketch]
  \uses{thm:6.3.18,lem:6.3.19}
  Follows directly from Theorem 6.3.18 and Lemma 6.3.19.
\end{proof}

\begin{corollary}[6.3.24]
  \label{cor:6.3.24}
  \uses{def:binary_matroid,def:isomorphism,def:minor,def:1_elem_ext,def:2_elem_ext,prop:6.3.12,prop:6.3.21,prop:6.3.22,prop:6.3.23,def:k_sep}
  Let $\M$ be a class of binary matroids closed under isomorphism and under taking minors. Suppose $N$ given by $B^{N}$ of (6.3.12) is in $\M$, but the $1$- and $2$-element extensions of $N$ given by (6.3.21), (6.3.22), (6.3.23), and by the accompanying conditions are not in $\M$. Assume matroid $M \in \M$ has an $N$ minor.
  Then any $k$-separation of any such minor that corresponds to $(X_{1} \cup Y_{1}, X_{2} \cup Y_{2})$ of $N$ under one of the isomorphisms induces a $k$-separation of $M$.
\end{corollary}

\begin{proof}[Proof sketch]
  \uses{thm:6.3.20}
  \begin{itemize}
    \item Let $M \in \M$ satisfying the assumptions. Since $\M$ is closed under isomorphism, suppose that $N$ itself is a minor of $M$.
    \item Suppose the $k$-separation of $N$ does not induce one in $M$. Then $M$ or a minor of $M$ containing $N$ is minimal under isomorphism.
    \item By Theorem 6.3.20, $M$ has a minor represented by (6.3.21), (6.3.22), or (6.3.23). This minor is in $\M$, as $\M$ is closed under taking minors, but this contradicts our assumptions.
  \end{itemize}
\end{proof}


\subsection{Chapter 6.4}

\begin{theorem}[6.4.1]
  \label{thm:6.4.1}
  \uses{def:k_conn,def:binary_matroid,def:minor,def:1_elem_ext,def:2_elem_ext}
  Let $M$ be a $3$-connected binary matroid with a $3$-connected proper minor $N$. Suppose $N$ has at least $6$ elements. Then $M$ has a $3$-connected minor $N'$ that is a $1$- or $2$-element extension of some $N$ minor of $M$. In the $2$-element case, $N'$ is derived from the $N$ minor by one addition and one expansion.
\end{theorem}

\begin{proof}[Proof sketch]
  \uses{lem:5.2.4,lem:6.2.6,thm:6.3.20}
  \begin{itemize}
    \item Let $z \in M \setminus N$. By Lemma 5.2.4, there is a connected minor $N'$ that is a $1$-element extension of $N$ by $z$. Our theorem holds iff it holds for duals, so by duality, assume that the extension is an addition.
    \item Reason about a matrix representation of $N$ and $N'$ to get a $2$-separation of $N'$. Since $M$ is $3$-connected, this $2$-separation does not induce one in $M$. Let $M'$ be a minor of $M$ that proves this fact and is minimal under isomorphism. Additionally, $M'$ has an $N'$ minor, so we change the element labels in $M'$ so that $N'$ is a minor of $M'$.
    \item Apply Theorem 6.3.20 and perform case analysis, reaching either a contradiction or a desired extension.
  \end{itemize}
\end{proof}
